%% Protokoll zur Erstellung und Handhabung des Fragebogen 2
%%	erstellt am: 13/07/24
 
\documentclass[a4paper]{scrartcl}
 
\usepackage[utf8]{inputenc}
\usepackage[T1]{fontenc}
\usepackage{foilhtml}
\usepackage[ngerman,english]{babel}
\usepackage{amsmath}
\usepackage{amsfonts}
\usepackage{hyperref}
\usepackage{listings}
\lstset{
	language=HTML,
	extendedchars=false,
	basicstyle=\ttfamily\footnotesize,
	tabsize=2,
	framesep=10pt,
	showstringspaces=false,
	fontadjust=true,
	flexiblecolumns=true
	}
  
\title{Dokumentation zum Fragebogen 2}
\author{Frank Baethge}
\date{2013-08-24}
\begin{document}
%%\maketitle
%%\tableofcontents
	\section{Programmierung}

		\subsection{TODO}
			\begin{itemize}
				\item UI

					\begin{itemize}
						\item teste übergebenen Parameter "'device"'
						\item Einstellungen programmieren (Starttag, Startzeiten, Gerätenamen)
						\item es sollen nicht immer alle Fragebögen (abhängig von der Zeit) auswählbar sein
						\item es sollen Haken gesetzt werden, damit man sieht welche Fragebögen bearbeitet wurden, am nächsten Tag sind die Haken wieder zu löschen, eine Routiene soll erkennen, ob alle Fragen eines Durchlaufs beantwortet sind, oder ob noch welche fehlen
						\item wenn die letzte Frage auf der Seite beantwortet ist auch ohne auf weiter zu klicken zur nächsten Seite gehen
						\item Header soll Fortschritt anzeigen
					\end{itemize}
				\item \href{http://developer.chrome.com/apps}{offline Applikation}\newline
					die app kann jederzeit geschlossen werden, sollte dort fortsetzen wo sie war\newline
					offline-app onLaunch(), onSuspend()\newline
					oder chrome://flags enable offline cache\newline
					oder \href{http://appcachefacts.info/}{appcache} - HTML5 - das scheint mir erstmal das Einfachste, das zum Ziel führen könnte. Offline-App ist bestimmt schicker, bedeutet aber auch, dass localStorage nicht verwendet werden kann. Das Verwenden von async bedeutet andererseits Beschleunigung des Codes.
					oder \href{https://developers.google.com/chrome/apps/docs/developers_guide?csw=1}{hosted app}
				\item DatenAblage auf Gerät
				\item Hinweis für Nutzer, dass ein Fragebogen ausgefüllt werden soll (Wecker evtl. mit Erinnerung)
				\item Datenübertragung auf Server
				\item Hosting auf UniMainz-Irgendwo: ssh, webserver + php, ajax (mysql, git, node.js oder modWebSocket)
				\item \href{http://mirror.math.ku.edu/tex-archive/macros/latex/contrib/listings/listings.pdf}{Listing anpassen - JavaScript} 
				\item http://tutorialzine.com/2013/06/digital-clock-adding-alarms/ 
			\end{itemize}

		\subsection{Informationen zum Programm}
			\begin{itemize}
				\item Testen mit: 192.168.178.20:8000
				\item Parameter: device=bla $\rightarrow$ Gerätename
				\item Parameter: st=jjjj-mm-tt $\rightarrow$ Startdatum (Tag)
				\item Parameter: sz=13:45 $\rightarrow$ Startzeit (Zeit des Arbeitsbeginns für jeden Tag)\newline
					Zeit und Tag überschreiben bereits vorhandene Zeit- und Tag-Werte
			\end{itemize}

		\subsection{DONE}
			\begin{itemize}
				\item google chrome beta installiert um Debugging zu ermöglichen siehe \href{https://developers.google.com/chrome-developer-tools/docs/remote-debugging}{remote-debugging}
				\item \href{http://developer.android.com/tools/device.html#setting-up}{udev einrichten} lege dazu /etc/udev/rules.d/51-android.rules neu an
				\item installiere adt (sdk und eclipse), weil irgendwas mit den Devices nicht zu funktionieren scheint \newline apt-get install lib32ncurses5 lib32stdc++6
				\item Domäne an Uni Mainz erhalten (SSH, PHP, MySQL, evtl. node.js)
				\item Anmeldungen für Google-Play (Chrome 28 - update)

					Vorname Name: 1 AOW \newline
					mail: AOWMainz@gmail.com - Kennwort in pw.gpg \newline
					bester Freund in der Kindheit: Kennwort in pw.gpg
				\item \href{http://www.ofbrooklyn.com/2012/11/13/backbonification-migrating-javascript-to-backbone}{Umwandlung einer bestehenden Seite}
				\item \href{http://backbonetutorials.com/organizing-backbone-using-modules/}{Organizing your app using Modules (require.js)}
				\item \href{http://developer.chrome.com/apps/about_apps.html}{Offline-App} als Chrome-Erweiterung
					
					\href{http://docs.webplatform.org/wiki/tutorials/offline_storage}{localStorage verwenden}\newline
					\href{http://docs.webplatform.org/wiki/tutorials/appcache_beginner}{application cache} mit manifest
				\item \href{http://jquerymobile.com/demos/1.3.0-rc.1/docs/lists/lists-themes.html}{data-icon} - alle Icons auf einen Blick
				\item 
			\end{itemize}
		\subsection{Geräte}
			\begin{itemize}
				\item 479006020ba8303a \newline
					IMEI:356098050204712/01 (mit *\#06\# zu erhalten) \newline
					WLAN:android-d98263c8b9c79f4f \newline
					WLAN-Mac:10:d5:42:46:b7:5b \newline
					Seriennummer: RF1D625WH0B
				\item  
					IMEI:358309055100419/01 \newline
					WLAN:android-d83bb82851286680 \newline
					WLAN-Mac:6c:b7:f4:a9:df:23 \newline
					Seriennummer: R21D80CMH2B
				\item Inbetriebnahme
					\begin{itemize}
						\item Sprache auswählen
						\item keine Sim-Karte $\rightarrow$ weiter
						\item WLAN einrichten: Netzwerk hinzufügen; Erweitert | WLAN im Standbymodus eingeschaltet lassen: nie
						\item Samsung-Konto: keins einrichten
						\item Google-GMail: aowmain.gmail.com -- vae2RudeiP\newline
							Update von Chrom-Browser
						\item eventuell weitere Software-Updates installieren (Neustart)
						\item Startbildschirm leeren und alle Seitenfenster löschen

					\end{itemize}
			\end{itemize}
		\section{Quellen}
			\subsection{index.html}
				\lstinputlisting{index.html}
			\subsection{fb2.css}
				\lstinputlisting{fb2.css}
			\subsection{fb2.appcache}
				\lstinputlisting{fb2.appcache}
%			\subsection{xxx}
%				\lstinputlisting{xxx}
			\subsection{Templates}
				\subsubsection{mt5A}
					\lstinputlisting{templates/mt5A.html}
				\subsubsection{mt5B}
					\lstinputlisting{templates/mt5B.html}
				\subsubsection{frage5}
					\lstinputlisting{templates/frage5.html}
				\subsection{frage7}
					\lstinputlisting{templates/frage7.html}
				\subsection{frage20}
					\lstinputlisting{templates/frage20.html}
				\subsection{fView}
					\lstinputlisting{templates/fView.html}
				\subsection{fehlerView}
					\lstinputlisting{templates/fehlerView.html}
				\subsubsection{settingsView}
					\lstinputlisting{templates/settingsView.html}
			\subsection{JavaScripte}
				\subsubsection{config}
					\lstinputlisting{js/config.js}
				\subsubsection{frage}
					\lstinputlisting{js/frage.js}
				\subsubsection{fragen}
					\lstinputlisting{js/fragen.js}
				\subsubsection{fb2Model}
					\lstinputlisting{js/fb2Model.js}
				\subsubsection{mtView}
					\lstinputlisting{js/mtView.js}
				\subsubsection{router}
					\lstinputlisting{js/router.js}
				\subsection{fb2Model}
					\lstinputlisting{js/fb2Model.js}
\end{document}
